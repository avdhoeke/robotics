% ------ PACKAGES ------

%\usepackage[cache=false]{minted}
\usepackage[utf8]{inputenc}%encodage en utf8
\usepackage[english]{babel} %permet caractères français
\usepackage[T1]{fontenc} %permet cararct spéciaux
\usepackage{hyperref}  %liens hypertextes
\hypersetup{
    colorlinks=true,
    linkcolor=blue,
    filecolor=magenta,      
    urlcolor=cyan,
}
\usepackage{cancel}
\usepackage[nottoc]{tocbibind}
\usepackage{csquotes}
\usepackage{tikz}
\usepackage{bm}
\usepackage{afterpage}
\usepackage{enumitem}
\usepackage{amsmath} %pour les maths
\usepackage{amsthm}
\usepackage{amssymb}
\usepackage[final]{pdfpages} %pdfinclud
\usepackage{acronym} %si on utilise de acronymes
\usepackage{setspace}%pour avoir un inteligne de 1.5
\usepackage{multirow}  %Tableaux à plusieurs lignes
\usepackage{geometry}%réglages mise en page
\usepackage{subcaption}
\usepackage{fancyhdr}%for headers and footers
\pagestyle{fancy}
\usepackage{amssymb}
\onehalfspacing   %interligne de 1.5 (demandé dans les consignes)
\usepackage{float} %pou placer les images à l'endroit dans le code
\usepackage{listingsutf8}
\usepackage{color} %red, green, blue, yellow, cyan, magenta, black, white
\definecolor{mygreen}{RGB}{28,172,0} % color values Red, Green, Blue
\definecolor{mylilas}{RGB}{170,55,241}
\usepackage{wrapfig}%permet figure format paysage
\usepackage{lscape}%idem
\usepackage{rotating}%idem
\usepackage{epstopdf}%idem
\usepackage{graphicx}%pour images
\usepackage{caption}%idem
\usepackage{subcaption}%idem
\lstset{basicstyle={\ttfamily\footnotesize},}% Pour inclure du code
\usepackage{bigcenter} %Pour centrer un grand tableau en comptant les marges
\usepackage{lscape} %Pour ecrire en format paysage
\usepackage{listings}
\usepackage{afterpage}
\usepackage[gobble=auto]{pythontex}
\newtheorem{definition}{Definition}
\usepackage{booktabs}
\usepackage{tabularx}
\usepackage{fourier} 
\usepackage{array}
\usepackage{makecell}
\usepackage{tikz}
\usetikzlibrary{positioning}

\usepackage{listings}

\definecolor{mygreen}{rgb}{0,0.6,0}
\definecolor{mygray}{rgb}{0.5,0.5,0.5}
\definecolor{mymauve}{rgb}{0.58,0,0.82}

\lstset{ %
  backgroundcolor=\color{white},   % choose the background color
  basicstyle=\footnotesize,        % size of fonts used for the code
  breaklines=true,                 % automatic line breaking only at whitespace
  captionpos=b,                    % sets the caption-position to bottom
  commentstyle=\color{mygreen},    % comment style
  escapeinside={\%*}{*)},          % if you want to add LaTeX within your code
  keywordstyle=\color{blue},       % keyword style
  stringstyle=\color{mymauve},     % string literal style
}

% Hrule command for cover page
\newcommand{\HRule}{\rule{\linewidth}{0.5mm}} %newcommand for cover page



